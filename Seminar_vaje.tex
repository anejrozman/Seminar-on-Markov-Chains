\documentclass[]{beamer}
\usepackage[T1]{fontenc}
\usepackage[utf8]{inputenc}
\usepackage[slovene]{babel}
\usepackage{pgfpages}
\usepackage{amsmath}
\usepackage{amssymb}
\usepackage{colortbl}
\usepackage{tikz}
\usetikzlibrary{positioning}
\usetikzlibrary{automata}
\usepackage{array}

\setbeameroption{hide notes}
%\setbeameroption{show only notes}                  
%\setbeameroption{show notes on second screen=right}

\mode<presentation>
\usetheme{Berlin}
\useinnertheme[shadows]{rounded}
\useoutertheme{infolines}
\usecolortheme{seahorse}
\usepackage{palatino}
\usefonttheme{serif}

\beamertemplatenavigationsymbolsempty
%\setbeamertemplate{headline}{}
%\setbeamertemplate{footline}{}
		
\title[Rešitve nalog]{Rešitve nalog za seminar}
\author[]{Anej Rozman, Jošt Plevel, Tanja Luštrek, Lucija Tekavc}
\institute[Mentor: doc. dr. Martin Raič]{Fakulteta za matematiko in fiziko}
\date[]{}

\begin{document}

\maketitle

\begin{frame}{1. Naloga}
    Mečemo pošten kovanec, meti so neodvisni. Za $n = 0, 1, 2, \ldots$ naj bo $X_n$ skupno število grbov v $(n + 1)$-tem in $(n + 2)$-tem metu (torej $0$, $1$ in $2$). Ali zaporedje $X_1, X_2, X_3, \ldots$ tvori Markovsko verigo?


\end{frame}



\begin{frame}{2. Naloga}

    Dana je Markovska veriga s prehodno matriko:
$$ p =
    \begin{bmatrix}
        0 & 0 & 0{,}1 & 0{,}9 \\
        0 & 0 & 0{,}6 & 0{,}4 \\
        0{,}8 & 0{,}2 & 0 & 0 \\
        0{,}4 & 0{,}6 & 0 & 0 \\
    \end{bmatrix}
$$       

a) Izračunajte vse stacionarne porazdelitve markovskih verig s prehodnima matrikama $p$ in $p^2$. \newline
b) Izračunajte $\lim_{n\to\infty} p^{2n}$

\end{frame}



\begin{frame}{3. Naloga}

    S police v čitalnici, na kateri je $n$ knjig, se $k$-ta knjiga vzame z verjetnostjo $p_k$. Ko
    obiskovalec pogleda knjigo, jo knjižničar vrne in postavi na najbolj levo mesto na polici,
    preostale knjige pa se ustrezno zamaknejo desno. Privzamemo, da se s police vsakič vzame
    le po ena knjiga in da so jemanja med seboj neodvisna. Tako dobimo markovsko verigo,
    katere stanja so permutacije množice ${1, 2, \ldots, n}$, torej zaporedja $(k_1, k_2, \ldots, k_n)$, kjer je
    $k_i$ številka knjige na $i$-tem mestu. Dokažite, da verjetnostna funkcija:

$$
\pi(k_1, k_2, \ldots, k_n) = p_{k_1} \cdot \dfrac{p_{k_2}}{1 - p_{k_1}} \cdot \dfrac{p_{k_3}}{1 - p_{k_1} - p_{k_2}} \cdots \dfrac{p_{k_n}}{1 - p_{k_1} - p_{k_2} - \ldots - p_{k_{n-1}}}
$$

predstavlja stacionarno porazdelitev te markovske verige.
\end{frame}






\end{document}